\fbox{
    \begin{minipage}[t]{0.31\textwidth}
        \scriptsize
        \textbf{Queue Operations}\\        
        \textbf{Queue-Empty(Q):}\\[2px]
        1. Returns TRUE if the queue is empty (Q.head = Q.tail).\\
        2. Returns FALSE otherwise.\\[2pt]
        \textbf{Enqueue(Q, x):}\\
        1. Adds element x to the rear of queue Q.\\
        2. Q[Q.tail] = x\\
        3. Q.tail = Q.tail + 1 (or wrap around if using circular array)\\[2pt]
        \textbf{Dequeue(Q):}\\
        1. If Queue-Empty(Q), return error "underflow".\\
        2. Otherwise, remove and return the element at the front.\\
        3. x = Q[Q.head]\\
        4. Q.head = Q.head + 1 (or wrap around)\\
        5. Return x\\[2pt]
        \textbf{Queue Implementation:}\\
        1. Q.head: Index of the front element\\
        2. Q.tail: Index where next element will be inserted\\
        3. In a circular array, indices wrap around\\
        4. Leave one slot empty to distinguish full/empty states

        \textit{Time Complexity:} \(O(1)\) for all operations \quad \textit{Space Complexity:} \(O(n)\)\\
    \end{minipage}
} 