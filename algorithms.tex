\documentclass[a4paper]{extarticle}
\usepackage[a4paper, top=0in, bottom=0in, left=0in, right=0in, includeheadfoot, includehead, includefoot]{geometry} % Minimal margins
% Force absolute minimal margins
\setlength{\hoffset}{-1in}
\setlength{\voffset}{-1in}
\setlength{\oddsidemargin}{0pt}
\setlength{\evensidemargin}{0pt}
\setlength{\topmargin}{0pt}
\setlength{\headheight}{0pt}
\setlength{\headsep}{0pt}
\setlength{\marginparwidth}{0pt}
\setlength{\marginparsep}{0pt}
\setlength{\footskip}{0pt}
\usepackage{amsmath,amssymb,amsthm}
\usepackage{enumitem}
\usepackage{graphicx}
\usepackage{ragged2e}
\usepackage[absolute,overlay]{textpos}
\usepackage{array}
\usepackage{booktabs}
\usepackage{multicol}
\usepackage{setspace}
\usepackage{fancyvrb}
\usepackage{titlesec}
\usepackage{xcolor}
\usepackage{microtype}
\usepackage{anyfontsize} % Allow any font size
\usepackage{mathptmx} % Georgia-like font (Times with math support)$\usepackage{algorithm}
\usepackage{algorithm}
\usepackage[noend]{algpseudocode}
\renewcommand{\familydefault}{\rmdefault} % Set Georgia as default font
\usepackage{paracol}
% Define custom font size command that works everywhere
\newcommand{\tinyfont}{\fontsize{5.2pt}{1pt}\selectfont}

% Define special ultra-small font for algorithms
\newcommand{\algofont}{\fontsize{4.3pt}{4.7pt}\selectfont}
\linespread{1.05}
% Set global font size
\tinyfont

% Make math fonts match text size
\DeclareMathSizes{0.8pt}{0.8pt}{0.6pt}{0.5pt}

% Make algorithm math fonts even smaller
\DeclareMathSizes{4pt}{4pt}{3pt}{2pt}

% Ensure minipages inherit the font size
\let\oldminipage\minipage
\renewcommand{\minipage}[2][]{\oldminipage[#1]{#2}\tinyfont}

% ──────────────────────────────────────────────────────────────
%  Algorithms (algorithm + algpseudocode)
% ──────────────────────────────────────────────────────────────
\usepackage{algorithm}
\usepackage[noend]{algpseudocode}

%----------------------- 1.  Appearance ------------------------
\AtBeginEnvironment{algorithm}{\tiny}        % smaller body text
\algrenewcommand\algorithmicindent{2pt}     % compact indent

% Right-flushed, grey comments with a ▹ bullet
\algrenewcommand\algorithmiccomment[1]{\textit{// #1}}

%----------------------- 2.  Keywords --------------------------
% lower-case keywords, remove redundant "then/end" words
\algrenewcommand\algorithmicif{\textbf{if}}
\algrenewcommand\algorithmicelse{\textbf{else}}
\algrenewcommand\algorithmicfor{\textbf{for}}
\algrenewcommand\algorithmicwhile{\textbf{while}}
\algrenewcommand\algorithmicreturn{\textbf{return}}
\algrenewcommand\algorithmicthen{}
\algrenewcommand\algorithmicend{}

% no "Function/Procedure" headings and a lower-case "call"
\algrenewcommand\algorithmicfunction{}
\algrenewcommand\algorithmicprocedure{}

%----------------------- 3.  < and > as math -------------------
% Make plain < and > automatically behave like math relations
\makeatletter
\newcommand{\algLt}{\ensuremath{<}}\newcommand{\algGt}{\ensuremath{>}}
\AtBeginEnvironment{algorithmic}{%
  \begingroup
    \lccode`\~=`\<
    \lowercase{\endgroup\def~{\algLt}}%
  \begingroup
    \lccode`\~=`\>
    \lowercase{\endgroup\def~{\algGt}}%
  \catcode`\<=\active \catcode`\>=\active
}
\makeatother

%----------------------- 4.  Remove algorithm rules -----------
% Remove the horizontal rules at top and bottom of algorithms
\usepackage{float}
\floatstyle{plain}
\restylefloat{algorithm}

%----------------------- 5.  Ultra-small loader ---------------
% Usage: \algocode{mysort}  ⇒ loads code/mysort.tex with special small font
\newcommand{\algocode}[1]{%
  \vspace{0pt}%
  \begin{algorithm}[H]
    \setlength{\abovecaptionskip}{0pt}%
    \setlength{\belowcaptionskip}{0pt}
    \setlength{\intextsep}{0pt}%
    \setlength{\textfloatsep}{0pt}
    \algofont  % Use special ultra-small algorithm font
    \input{code/#1.tex}%
  \end{algorithm}%
  \vspace{0pt}%
}

% ─── End of algorithm setup ───────────────────────────────────


% Additional margin elimination
\setlength{\parindent}{0.5pt}
\setlength{\parskip}{0.5pt}
\setlength{\baselineskip}{0pt}
\pagestyle{empty}

% Fix fbox overlapping issues
\setlength{\fboxrule}{0.4pt}  % Thinner border
\setlength{\fboxsep}{1pt}     % Less padding inside box

% Maximize text width
\setlength{\textwidth}{\paperwidth}
\setlength{\textheight}{\paperheight}

% Remove default spacing from center environment
\let\oldcenter\center
\let\endoldcenter\endcenter
\renewenvironment{center}{\oldcenter\vspace{-\topsep}\vspace{-\partopsep}}{\vspace{-\topsep}\vspace{-\partopsep}\endoldcenter}

\begin{document}

%%%%%%%%%%%%%%%%%%%%%%%%%%%%%%%%%%%%%%%%%%%%%%
% Title page
\begin{titlepage}
    \centering
    \vspace*{0.5cm}
    {\Huge \textbf{Algorithms - CheatSheet}}\\[10px] 
    {\LARGE IN~BA4 - Ola Nils Anders Svensson}\\[10px]
    {\Large Notes by Ali EL AZDI}\\[10px]
    \vfill
    \begin{minipage}{0.8\textwidth}
        \begin{justify}
            \textit{This is a cheat sheet for the Algorithms midterm exam. For suggestions, contact me on Telegram (\texttt{elazdi\_al}) or via EPFL email (\texttt{ali.elazdi@epfl.ch}).}
        \end{justify}
    \end{minipage}\\
    \vspace{30px}
    {\large March 25th, 2025}
    \vspace{40px}
\end{titlepage} 
\newpage
%%%%%%%%%%%%%%%%%%%%%%%%%%%%%%%%%%%%%%%%%%%%%%
\vspace*{-15px}
\noindent
   \columnratio{0.31,0.28,0.35}
\begin{paracol}{3}
\setlength{\columnsep}{3pt}
\fbox{
\begin{minipage}[t]{1\linewidth}
\textbf{Master Theorem}\\
If $T(n) =aT\left(\frac{n}{b}\right) + f(n)$, $a \geq 1$, $b > 1$, and $f(n)$ asymptotically positive.\\
\textbf{Case 1: }If $f(n) = O(n^{\log_b a - \epsilon})$ for some $\epsilon > 0$, then $T(n) = \Theta(n^{\log_b a})$.\\
\textbf{Case 2: }If $f(n) = \Theta(n^{\log_b a})$, then $T(n) = \Theta(n^{\log_b a}\log n)$.\\
\textbf{Case 3: }If $f(n) = \Omega(n^{\log_b a + \epsilon})$ for some $\epsilon > 0$, and if $a\, f\left(\frac{n}{b}\right) \leq c\, f(n)$ for some $c < 1$ and all sufficiently large $n$, then $T(n) = \Theta(f(n))$.\\
\noindent
    \begin{center}
    \textbf{Common case - if $f(n) = \Theta(n^d)$ for some exponent $d$:}\\[0.3em]
    \end{center}
- If $\frac{a}{b^d} < 1$ (or $d > \log_b a$), then $T(n) = \Theta(n^d)$.\\
- If $\frac{a}{b^d} = 1$ (or $d = \log_b a$), then $T(n) = \Theta(n^d \log n)$.\\
- If $\frac{a}{b^d} > 1$ (or $d < \log_b a$), then $T(n) = \Theta(n^{\log_b a})$.
    \end{minipage}
    }\\
  \fbox{
      \begin{minipage}[t]{1\linewidth}
          \textbf{Queue Operations}\\        
          \textbf{Queue-Empty(Q):} \textit{Time: O(1), Auxiliary Space: O(1)}\\
          1. Returns TRUE if the queue is empty (Q.head = Q.tail).\\
          2. Returns FALSE otherwise.\\
          \textbf{Enqueue(Q, x):} \textit{Time: O(1), Auxiliary Space: O(1)}\\
          1. Adds element x to the rear of queue Q.\\
          2. Q[Q.tail] = x\\
          3. Q.tail = Q.tail + 1 (or wrap around if using circular array)\\
          \textbf{Dequeue(Q):} \textit{Time: O(1), Auxiliary Space: O(1)}\\
          1. If Queue-Empty(Q), return error "underflow".\\
          2. Otherwise, remove and return the element at the front.\\
          3. x = Q[Q.head]\\
          4. Q.head = Q.head + 1 (or wrap around)\\
          5. Return x\\
          \textbf{Queue Implementation:}\\
          1. Q.head: Index of the front element\\
          2. Q.tail: Index where next element will be inserted\\
          3. In a circular array, indices wrap around\\
          4. Leave one slot empty to distinguish full/empty states
  
          \textit{Overall Space Complexity:} \(O(n)\) for a queue of capacity n
      \end{minipage}
  }

  \fbox{
  \begin{minipage}[t]{1\linewidth}
    \textbf{Merge Sort}\\
      \begin{minipage}[t]{0.6\textwidth}
   1. \textbf{Divide}: Split the array evenly into two smaller subarrays, and continue dividing recursively.\\
   2. \textbf{Sort (Recursively)}: Apply merge sort recursively on each subarray until each has only one element (base case).\\[-10px]
   \begin{minipage}[t]{0.6\textwidth}
   \algocode{merge-sort}
   \end{minipage}\\[-22px]
   3. \textbf{Merge}: Combine the two sorted subarrays into a single sorted array:
          \begin{enumerate}[noitemsep, topsep=0pt, leftmargin=3pt, rightmargin=0pt]
              \item[]a) Initializing pointers at the start of each subarray.
              \item[]b) Comparing the elements pointed to, and appending the smaller one into a new array.
              \item[]c) Advancing the pointer in the subarray from which the element was chosen.
              \item[]d) Repeating this process until all elements in both subarrays are merged into the sorted array.
          \end{enumerate}
      \textit{Merge Cost Complexity:} \(O(n)\) per merge operation.\\
  \textit{Time Complexity:} \(O(n \log n)\) \\ \textit{Space Complexity:} \(O(n)\)
      \end{minipage}
      \hfill
      \begin{minipage}[t]{0.39\textwidth}
    \algocode{merge}
      \end{minipage}
      \vspace*{-10px}
      \end{minipage}
  }

  \fbox{
     \begin{minipage}[t]{1\linewidth}{
      \textbf{Priority Queue}\\
      Maintains a dynamic set of elements with associated priority values (keys).\\
      
      \textbf{Maximum(S):} Return element of S with highest priority (return A[1], complexity $O(1)$)\\
      
      \textbf{Insert(S,x):} Insert element x into set S\\
      1. Increment the heap size\\
      2. Insert a new node in the last position in the heap, with key $-\infty$\\
      3. Increase the $-\infty$ value to key using Heap-Increase-Key\\
      
      \textbf{Extract-Max(S):} Remove and return element of S with highest priority\\
      1. Make sure heap is not empty\\
      2. Make a copy of the maximum element (the root)\\
      3. Make the last node in the tree the new root\\
      4. Re-heapify the heap, with one fewer node\\
      5. Return the copy of the maximum element\\
      
      \textbf{Increase-Key(S,x,k):} Increase the value of element x's key to the new value k\\
      1. Make sure key $\geq$ A[i]\\
      2. Update A[i]'s value to key\\
      3. Traverse the tree upward comparing new key to the parent and swapping if necessary\\
      \textit{Time Complexity:} Insert, Extract-Max, Increase-Key: $O(\log n)$ \quad Maximum: $O(1)$
      \textit{Space Complexity:} $O(n)$}
     \end{minipage}
  }\\
  \fbox{
  \begin{minipage}[t]{1\linewidth}
    \textbf{Stack Operations}\\
    \textbf{Stack-Empty(S):} \textit{Time: O(1), Auxiliary Space: O(1)}\\[1pt]
    1. Returns TRUE if the stack is empty.\\
    2. Returns FALSE otherwise.\\[1pt]
    \textbf{Push(S, x):} \textit{Time: O(1), Auxiliary Space: O(1)}\\
    1. Adds element x to the top of stack S.\\
    2. Increments the stack pointer.\\[1pt]
    \textbf{Pop(S):} \textit{Time: O(1), Auxiliary Space: O(1)}\\
    1. If Stack-Empty(S), return error "underflow".\\
    2. Otherwise, remove and return the top element.\\
    3. Decrements the stack pointer.\\[1pt]
    \textbf{Stack Implementation:}\\
    1. Elements are stored in a simple array\\
    2. S.top: Index of the topmost element\\
    3. An empty stack has S.top = 0 or S.top = -1 (implementation dependent)\\
    \textit{Overall Space Complexity:} \(O(n)\) for a stack of size n
  \end{minipage}
  }\\
  \fbox{
    \begin{minipage}[t]{1\linewidth}
      \textbf{Dynamic Programming}\\
      \begin{minipage}[t]{0.49\linewidth}
    \textbf{Problem:} Optimal solutions to overlapping subproblems\\
    \textbf{Fibonacci Sequence:}\\
    \textbf{Top-Down (Memoization):}\\
    1. Create memo array $F[0...n]$ initialized to NIL\\
    2. Base cases: $F[0] = 0$, $F[1] = 1$\\
    3. Recursive with memo:\\
    - Return $F[n]$ if already computed\\
    - Otherwise compute \\$F[n] = F[n-1] + F[n-2]$\\
    - Store result in $F[n]$ and return\\[-15px]
    \algocode{fib-memoized}\vspace*{-25px}
    \textbf{Bottom-Up (Tabulation):}\\
    1. Create array $F[0...n]$\\
    2. Base cases: $F[0] = 0$, $F[1] = 1$\\
    3. For $i = 2$ to $n$:\\
    - Compute $F[i] = F[i-1] + F[i-2]$\\
    4. Return $F[n]$\\[-15px]
    \algocode{fib-bottom-up}\vspace*{-20px}
    \textbf{Time Complexity:} $O(n)$\\
    \textbf{Space Complexity:} $O(n)$ \\(includes input and memo array)
    \end{minipage}
    \hfill
    \begin{minipage}[t]{0.49\linewidth}
    \textbf{Cut-Rod Problem:}\\
    Find optimal way to cut rod to maximize revenue\\
    \textbf{Top-Down (Memoization):}\\
    1. Create memo array $r[0...n]$ with $r[0] = 0$\\
    2. For uncalculated $r[j]$, compute:\\
    $r[j] = \max_{1\leq i\leq j}(p[i] + r[j-i])$\\
    3. Return $r[n]$\\[-17px]
    \algocode{rod-memoized}\vspace*{-25px}
    \textbf{Bottom-Up (Tabulation):}\\
    1. Create array $r[0...n]$ with $r[0] = 0$\\
    2. For $j = 1$ to $n$:\\
    - Compute $r[j] = \max_{1\leq i\leq j}(p[i] + r[j-i])$\\
    3. Return $r[n]$\\[-13px]
    \algocode{rod-bottom-up}\vspace*{-25px}
    \textbf{Time Complexity:} $O(n^2)$\\
    \textbf{Space Complexity:} $O(n)$ \\(includes input prices and array)
    \end{minipage}
    \end{minipage}
    }
  \switchcolumn[1]
  \fbox{
    \begin{minipage}[t]{1\linewidth}
        \textbf{Akra--Bazzi, Ali Najib Variation:}\\
        For recurrence $T(n)=\alpha\,T(a\,n)+\beta\,T(b\,n)+\Theta(n^d),$\\where $\alpha,\beta>0$, $a,b\in(0,1)$, $d\ge0$, and unique $p$ with $\alpha\,a^p+\beta\,b^p=1$\\
        Then:
        \begin{center}
        $p>d \Rightarrow T(n)=\Theta(n^p),$\\
        $p=d \Rightarrow T(n)=\Theta(n^d\log n),$\\
        $p<d \Rightarrow T(n)=\Theta(n^d).$\\[2px]
        \end{center}
        \textbf{If $p$ is not easily found, compare $\alpha\,a^d+\beta\,b^d$ with 1:}\\[2px]
        \begin{center}
        $\alpha\,a^d+\beta\,b^d<1 \Rightarrow T(n)=\Theta(n^d),$\\
        $\alpha\,a^d+\beta\,b^d=1 \Rightarrow T(n)=\Theta(n^d\log n),$\\
        $\alpha\,a^d+\beta\,b^d>1 \Rightarrow T(n)=\Theta(n^p),$\\[2px]
        \end{center}
        with $p$ determined by $\alpha\,a^p+\beta\,b^p=1$.
        \end{minipage}}\\
  \fbox{%
\begin{minipage}[t]{1.02\linewidth}
\textbf{Insertion Sort.}\\[-10px]
\begin{minipage}[htp]{0.48\linewidth}
1 - \textbf{Select the key}\\ 
Begin with the second element (at index 1) as the \textit{key}.\\[1pt]
2 - \textbf{Compare and Shift}\\ 
Compare the key with elements in the sorted section (to its left).\\[1pt]
3 - \textbf{Shift Elements}\\ 
If an element is greater than the key, shift that element one position to the right.
\end{minipage}%
\hfill
\begin{minipage}[htp]{0.52\linewidth}
\algocode{insertion-sort}
\end{minipage}\\[-10px]
4 - \textbf{Insert the Key}\\
Once an element less than or equal to the key is found (or you reach the start), insert the key immediately after that element.\\
5 - \textbf{Repeat}\\ 
Move forward to the next element, treating it as the new key, and repeat until the array is sorted.\\
\textit{Time Complexity:} Worst-case $O(n^2)$, Best-case $O(n)$.\\
\textit{Space Complexity:} $O(1)$.
\end{minipage}%
}\\
\fbox{
\begin{minipage}[t]{1\linewidth}
\textbf{Strassen's Matrix Multiplication}\\
 \textbf{Divide:} Partition each of \(A, B, C\) into four \(\tfrac{n}{2} \times \tfrac{n}{2}\) submatrices:\\
$
        \begin{pmatrix}
            C_{11} & C_{12} \\
            C_{21} & C_{22}
        \end{pmatrix}
        \;=\;
        \begin{pmatrix}
            A_{11} & A_{12} \\
            A_{21} & A_{22}
        \end{pmatrix}
        \cdot
        \begin{pmatrix}
            B_{11} & B_{12} \\
            B_{21} & B_{22}
        \end{pmatrix}.
$\\[2pt]
\begin{minipage}[t]{0.47\textwidth}
    \textbf{Conquer:} Compute 7 products (recursively on \(\tfrac{n}{2} \times \tfrac{n}{2}\) matrices):\\
$
        \begin{aligned}
        M_1 &:= (A_{11} + A_{22})(B_{11} + B_{22}), \\
        M_2 &:= (A_{21} + A_{22}) \, B_{11}, \\
        M_3 &:= A_{11} \,(B_{12} - B_{22}), \\
        M_4 &:= A_{22} \,(B_{21} - B_{11}), \\
        M_5 &:= (A_{11} + A_{12}) \, B_{22}, \\
        M_6 &:= (A_{21} - A_{11}) \,(B_{11} + B_{12}), \\
        M_7 &:= (A_{12} - A_{22}) \,(B_{21} + B_{22}).
        \end{aligned}
$
\end{minipage}
\hfill
\begin{minipage}[t]{0.5\textwidth}
    \textbf{Combine:} Assemble the resulting submatrices to form \(C\): \\[2px]
$
        \begin{aligned}
        C_{11} &= M_1 + M_4 - M_5 + M_7, \\
        C_{21} &= M_2 + M_4, \\
        C_{12} &= M_3 + M_5, \\
        C_{22} &= M_1 + M_3 - M_2 + M_6.
        \end{aligned}
$\\[2px]
\textit{Time Complexity:} $O(n^{\log_2 7}) \approx O(n^{2.81})$\\
\textit{Space Complexity:}$O(n^2)$
        \end{minipage}
\end{minipage}
}
\\
\fbox{\begin{minipage}[t]{1\linewidth}
   \textbf{Binary Search Trees (BST)} \\   
   \begin{minipage}[t]{0.49\linewidth}
       \textbf{BST-Search}\\
       \begin{minipage}[t]{\linewidth}
           1. Start at root\\
           2. If NULL, return NULL\\
           3. If key = root's key, return root\\
           4. If key $<$ root's key, search left\\
           5. If key $>$ root's key, search right
       \end{minipage}\\[-10px]
      \algocode{bst-search}\vspace*{-25px}
       \textit{Time:} \(O(\log n)\) avg, \(O(h)\) worst\\
       \textit{Space:} \(O(n)\) for tree, \(O(h)\) auxiliary\\
       \textbf{BST-Minimum}\\
       \begin{minipage}[t]{\linewidth}
           1. Start at root\\
           2. If NULL, return NULL\\
           3. Follow left pointers until no left child\\
           4. Return leftmost node
       \end{minipage}\\[-10px]
       \algocode{bst-minimum}\vspace*{-25px}
       \textit{Time:} \(O(h)\)\\
       \textit{Space:} \(O(n)\) for tree, \(O(1)\) auxiliary\\
       \textbf{BST-Maximum}\\
       \begin{minipage}[t]{\linewidth}
           1. Start at root\\
           2. If NULL, return NULL\\
           3. Follow right pointers until no right child\\
           4. Return rightmost node
       \end{minipage}\\[-10px]
       \algocode{bst-maximum}\vspace*{-25px}
       \textit{Time:} \(O(h)\)\\
       \textit{Space:} \(O(n)\) for tree, \(O(1)\) auxiliary\\
       \textbf{BST-Successor}\\
       \begin{minipage}[t]{\linewidth}
           1. If right subtree exists:\\
           \quad Return minimum in right subtree\\
           2. Otherwise:\\
           \quad Find first ancestor where\\
           \quad node is in left subtree
       \end{minipage}\\[-10px]
       \algocode{bst-successor}\vspace*{-25px}
       \textit{Time:} \(O(h)\)\\
       \textit{Space:} \(O(n)\) for tree, \(O(1)\) auxiliary\\
       \textbf{BST-Insert}\\
       \begin{minipage}[t]{\linewidth}
           1. Create new node z with key\\
           2. Start at root, track parent y = NIL\\
           3. Move down tree (left if key < node key, right otherwise)\\
           4. Once NULL found, link z as child of y\\
           5. If y is NIL, z becomes root\\
           6. Otherwise, insert z as left or right child based on key comparison
       \end{minipage}\\[-10px]
       \algocode{bst-insert}\vspace*{-25px}
       \textit{Time:} \(O(h)\)\\
       \textit{Space:} \(O(n)\) for tree, \(O(1)\) auxiliary
   \end{minipage}
   \hfill
   \begin{minipage}[t]{0.5\linewidth}
       \textbf{BST-Postorder}\\
       \begin{minipage}[t]{\linewidth}
           1. Recursively traverse left\\
           2. Recursively traverse right\\
           3. Visit current node\\
           \textit{(Children first, then root)}
       \end{minipage}\\[-10px]
       \algocode{bst-postorder}\vspace*{-25px}
       \textit{Time:} \(O(n)\)\\
       \textit{Space:} \(O(n)\) for tree, \(O(h)\) auxiliary\\
       \textbf{BST-Inorder}\\
       \begin{minipage}[t]{\linewidth}
           1. Recursively traverse left\\
           2. Visit current node\\
           3. Recursively traverse right\\
           \textit{(Visits nodes in sorted order)}
       \end{minipage}\\[-10px]
       \algocode{bst-inorder}\vspace*{-25px}
       \textit{Time:} \(O(n)\)\\
       \textit{Space:} \(O(n)\) for tree, \(O(h)\) auxiliary\\
       \textbf{BST-Preorder}\\
       \begin{minipage}[t]{\linewidth}
           1. Visit current node\\
           2. Recursively traverse left\\
           3. Recursively traverse right\\
           \textit{(Root first, then children)}
       \end{minipage}\\[-10px]
       \algocode{bst-preorder}\vspace*{-25px}
       \textit{Time:} \(O(n)\)\\
       \textit{Space:} \(O(n)\) for tree, \(O(h)\) auxiliary\\
       \textbf{BST-Transplant}\\
       \begin{minipage}[t]{\linewidth}
           Replace subtree at u with v:\\
           1. If u is root, set v as root\\
           2. If u is left child, make v left child of u's parent\\
           3. Else make v right child of u's parent\\
           4. Set v's parent to u's parent
       \end{minipage}\\[-10px]
       \algocode{bst-transplant}\vspace*{-25px}
       \textit{Time:} \(O(1)\)\\
       \textit{Space:} \(O(n)\) for tree, \(O(1)\) auxiliary\\
       \textbf{BST-Delete}\\
       \begin{minipage}[t]{\linewidth}
           1. If z has no left: transplant right\\
           2. If z has no right: transplant left\\
           3. With both children:\\
           \quad a. Find successor y\\
           \quad b. Handle y's children\\
           \quad c. Replace z with y
       \end{minipage}\\[-10px]
       \algocode{bst-delete}\vspace*{-25px}
       \textit{Time:} \(O(h)\)\\
       \textit{Space:} \(O(n)\) for tree, \(O(1)\) auxiliary
   \end{minipage}
   \vspace{2pt}
       \hrule
       \vspace{2pt}
       \textbf{Properties:}\\
- Left subtree: all keys < node's key\\
- Right subtree: all keys > node's key\\
- Left and right subtrees are also BSTs\\
- A \textbf{Node} has: \textbf{key} (value), \textbf{left} \& \textbf{right} (child pointers), \textbf{parent} (optional)\\
- Tree height \textbf{h}: length of longest path from root to leaf\\
\end{minipage}} 
  \switchcolumn[2]
\fbox{
  \begin{minipage}[t]{1\linewidth}
  \textbf{Big-O}\\
  If $\exists c>0$ and $\exists n_0>0$, $0\le f(n)\le c\cdot g(n)$ $\forall n\ge n_0$, $f(n)=O(g(n))$.\\
  \textbf{Big-Omega}\\
  If $\exists c>0$ and $\exists n_0>0$, $0\le c\cdot g(n)\le f(n)$ $\forall n\ge n_0$, $f(n)=\Omega(g(n))$.\\
  \textbf{Big-Theta}\\
  If $f(n)=O(g(n))$ and $f(n)=\Omega(g(n))$, $f(n)=\Theta(g(n))$.\\
  \textbf{Little-o}\\
  If $\forall c>0$ $\exists n_0>0$, $0\le f(n)< c\cdot g(n)$ $\forall n\ge n_0$, $f(n)=o(g(n))$.\\
  \textbf{Relations}\\
  $f(n)=o(g(n)) \implies f(n)=O(g(n))$\\
  \textbf{Comparison of Common Functions (Ascending Order)}\\
  $O(1) \ll O\bigl((\log n)^c\bigr) \ll O\bigl(n^c\bigr)_{0 < c < 1} \ll O(n) \ll O\bigl(n \log n\bigr) \ll O\bigl(n^c\bigr)_{c > 1} \ll O(c^n) \ll O(n!) \ll O(n^n).$
  \end{minipage}
  }  \\
  \fbox{
  \begin{minipage}[t]{1\linewidth}
    \textbf{Heap}  \fbox{\begin{minipage}[t]{0.55\linewidth}
    Root is A[1] \;
    Left(i) = 2i \; 
    Right(i) = 2i + 1 \;
    Parent(i) =$\lfloor i/2 \rfloor$\end{minipage}}\\
    \begin{minipage}[t]{0.67\textwidth}
          \textbf{ Max-Heapify (heapify subtree rooted at i)}\\
          1. Starting at the root\\
          2. Compare A[i], A[Left(i)], A[Right(i)]\\
          3. If necessary, swap A[i] with the largest of the two children\\
          4. \textbf{Max-Heapify} the swapped child\\
          5. Continue comparing and swapping down the heap until subtree rooted at i is max-heap\\
          \textit{Time Complexity:} \(O(\log(n))\) \\ \textit{Space Complexity:} \(O(n)\) including heap array, \(O(1)\) auxiliary
    \end{minipage}
    \begin{minipage}[t]{0.32\textwidth}
      \vspace*{-25px}
      \algocode{heapify}
      \vspace*{-25px}
    \end{minipage}\\
    \begin{minipage}[t]{0.67\textwidth}   
          \textbf{ Max-Heap-Insert (insert new key into heap)}\\
          1. Increase heap size: A.heap-size = A.heap-size + 1\\
          2. Set the last element to negative infinity: A[A.heap-size] = $-\infty$\\
          3. Call Max-Heap-Increase-Key to update to the correct value\\
          \textit{Time Complexity:} \(O(\log n)\) \\ \textit{Space Complexity:} \(O(n)\) including heap array, \(O(1)\) auxiliary
    \end{minipage}
    \begin{minipage}[t]{0.32\textwidth}
      \vspace*{-16px}
      \algocode{heap-insert}
      \vspace*{-15px}
    \end{minipage}\\
    \begin{minipage}[t]{0.67\textwidth}
          \textbf{ Max-Heap-Increase-Key (increase key at position i)}\\
          1. Ensure new key is larger than current: if key $<$ A[i] then error\\
          2. Set A[i] = key\\
          3. Compare with parent and swap if necessary: \\ while i > 1 and A[Parent(i)] < A[i]\\
          4. Exchange A[i] with A[Parent(i)]\\
          5. Set i = Parent(i) and continue upward\\
          \textit{Time Complexity:} \(O(\log n)\) \\ \textit{Space Complexity:} \(O(n)\) including heap array, \(O(1)\) auxiliary
    \end{minipage}
    \begin{minipage}[t]{0.32\textwidth}
      \vspace*{-16px}
      \algocode{heap-increase-key}
      \vspace*{-15px}
    \end{minipage}\\
   \begin{minipage}[t]{0.67\textwidth}
          \noindent \textbf{ Build-Max-Heap (build a max-heap from an array)}\\
          1. Start from the last non-leaf node at index \(\frac{n}{2} - 1\)\\
          2. Move upwards to the root (index \(0\)) and:\\
          \\ a. \textbf{Max-Heapify} the current node\\
          \\ b. Ensure the subtree rooted here satisfies max-heap property\\
          3. Repeat until the root node is processed\\
          4. After completion, array \(A\) represents a valid max heap\\
          \textit{Time Complexity:} \(O(n)\) \\ \textit{Space Complexity:} \(O(n)\) including heap array, \(O(1)\) auxiliary\\
          \textbf{ Heap Sort}\\
          \textbf{1. Build a Max Heap:}\\
          a. Convert the given array into a max heap\\
          b. Start from the last non-leaf node and heapify upwards\\
          c. Ensure each parent node is greater than its children\\
          \textbf{2. Extract Maximum Elements:}\\
          a. Swap the root (maximum value) with the last element\\
          b. Reduce heap size by one to exclude the last element\\
          c. Heapify the root to maintain max heap property\\
          d. Repeat until heap size becomes 1\\
          \textbf{3. Final Sorted Array:}\\
          a. After extraction, the sorted array in ascending order is obtained\\
          b. Maximum elements are placed at the end\\
          \textit{Time Complexity:} \(O(n\log n)\) \\ \textit{Space Complexity:} \(O(n)\) including array, \(O(1)\) auxiliary
      \end{minipage}
      \hfill
      \begin{minipage}[t]{0.32\textwidth}
        \vspace*{-15px}
        \algocode{build-max-heap}
        \algocode{heapsort}\vspace*{-30px}
        \algocode{heap-extract-max}\vspace*{-30px}
    \end{minipage}
    \end{minipage}
  }\\  \fbox{%
  \begin{minipage}[t]{0.44\textwidth}
  \begin{minipage}[t]{0.4\textwidth}
  \textbf{Maximum Subarray Problem}\\
  \textbf{Problem:} Find contiguous subarray with largest sum\\
  \textbf{Divide and Conquer Approach:}\\
  1 - \textbf{Divide.} \\
  Split array at midpoint\\$\text{mid} = \lfloor(\text{low} + \text{high})/2\rfloor$.\\
  2 - \textbf{Conquer.} \\
  Find maximum subarrays recursively
  \begin{itemize}[leftmargin=0pt,itemsep=0pt,topsep=0pt,parsep=0pt]
  \item[] Left max: in $A[\text{low}...\text{mid}]$
  \item[] Right max: in $A[\text{mid}+1...\text{high}]$
  \item[] Crossing max: spans the midpoint
  \end{itemize}
  3 - \textbf{Combine:} Return the largest of the three\\
  $\max(\text{left\_max}, \text{right\_max}, \text{crossing\_max})$\\
  \end{minipage}%
  \begin{minipage}[t]{0.56\textwidth}
    \vspace{-15px}
      \algocode{max-subarray}\vspace*{-35px}
  \end{minipage}\\
  \begin{minipage}[t]{0.5\textwidth}
  4 - \textbf{Finding the Crossing Maximum:}\\
  a. Find maximum suffix in left half (from mid down to low)\\
  b. Find maximum prefix in right half (from mid+1 up to high)\\
  c. Crossing max = max suffix + max prefix\\[1pt]
  \textbf{Crossing Subarray:} The maximum subarray that crosses the midpoint combines the maximum suffix of the left half with the maximum prefix of the right half.\\[2pt]
  \textit{Time Complexity:} $\Theta(n\log n)$ due to $T(n) = 2T(n/2) + \Theta(n)$\\
  \textit{Space Complexity:} $O(\log n)$ for recursion stack\\[2pt]
  \end{minipage}
  \begin{minipage}[t]{0.5\textwidth}
    \vspace{-20px}
  \algocode{max-subarray-crossing}
  \end{minipage}\vspace*{-23px}
  \end{minipage}%
  }\\
  \fbox{
    \begin{minipage}[t]{1\linewidth}
    \textbf{Data Structure Operations Summary}\\
    \begin{minipage}[t]{1\linewidth}
    \textbf{Priority Queue Operations}\\
    \begin{tabular}{|p{0.19\linewidth}|p{0.37\linewidth}|p{0.168\linewidth}|p{0.04\linewidth}|}
    \hline
    \textbf{Operation} & \textbf{Description} & \textbf{Time} & \textbf{Space} \\
    \hline
    Construction & Create priority queue from array of n elements & $O(n)$ & $O(n)$ \\
    \hline
    Maximum(S) & Returns element with highest priority & $O(1)$ & $O(1)$ \\
    \hline
    Extract-Max(S) & Removes and returns element with highest priority & $O(\log n)$ & $O(1)$ \\
    \hline
    Insert(S, x) & Inserts element x into set S & $O(\log n)$ & $O(1)$ \\
    \hline
    Increase-Key(S, x, k) & Increases priority of element x to k & $O(\log n)$ & $O(1)$ \\
    \hline
    \end{tabular}\\
    \textbf{Stack Operations}\\
    \begin{tabular}{|p{0.19\linewidth}|p{0.37\linewidth}|p{0.168\linewidth}|p{0.04\linewidth}|}
    \hline
    \textbf{Operation} & \textbf{Description} & \textbf{Time} & \textbf{Space} \\
    \hline
    Construction & Create stack from array of n elements & $O(n)$ & $O(n)$ \\
    \hline
    Stack-Empty(S) & Returns TRUE if stack is empty, FALSE otherwise & $O(1)$ & $O(1)$ \\
    \hline
    Push(S, x) & Adds element x to top of stack S & $O(1)$ & $O(1)$ \\
    \hline
    Pop(S) & Removes and returns top element from stack S & $O(1)$ & $O(1)$ \\
    \hline
    \end{tabular}\\
    \textbf{Heap Operations}\\
    \begin{tabular}{|p{0.19\linewidth}|p{0.37\linewidth}|p{0.168\linewidth}|p{0.04\linewidth}|}
    \hline
    \textbf{Operation} & \textbf{Description} & \textbf{Time} & \textbf{Space} \\
    \hline
    Construction & Create heap from array of n elements & $O(n)$ & $O(n)$ \\
    \hline
    Max-Heapify(A, i) & Maintains max-heap property at node i & $O(\log n)$ & $O(1)$ \\
    \hline
    Build-Max-Heap(A,n) & Converts array A of n elements into max heap & $O(n)$ & $O(1)$ \\
    \hline
    Heap-Sort(A, n) & Sorts array A of n elements using heap structure & $O(n \log n)$ & $O(1)$ \\
    \hline
    Max-Heap-Insert(A, k) & Inserts key k into heap A & $O(\log n)$ & $O(1)$ \\
    \hline
    Heap-Extract-Max(A) & Returns and removes largest element from heap A & $O(\log n)$ & $O(1)$ \\
    \hline
    Heap-Increase-Key(A,i,k) & Increases key at index i to new value k & $O(\log n)$ & $O(1)$ \\
    \hline
    \end{tabular}\\
    \end{minipage}\\
    \begin{minipage}[t]{1\linewidth}
    \textbf{BST Operations}\\
    \begin{tabular}{|p{0.19\linewidth}|p{0.37\linewidth}|p{0.168\linewidth}|p{0.04\linewidth}|}
    \hline
    \textbf{Operation} & \textbf{Description} & \textbf{Time} & \textbf{Space} \\
    \hline
    Construction & Create BST from array of n elements & Best: $O(n\log n)$, Worst: $O(n^2)$ & $O(n)$ \\
    \hline
    BST-Search(T, k) & Finds node with key k in tree T & $O(h)$ & $O(h)$ \\
    \hline
    BST-Minimum(T) & Returns node with smallest key in T & $O(h)$ & $O(1)$ \\
    \hline
    BST-Maximum(T) & Returns node with largest key in T & $O(h)$ & $O(1)$ \\
    \hline
    BST-Successor(x) & Returns node with smallest key greater than x's key & $O(h)$ & $O(1)$ \\
    \hline
    BST-Insert(T, z) & Inserts node z into BST T & $O(h)$ & $O(1)$ \\
    \hline
    BST-Delete(T, z) & Removes node z from BST T & $O(h)$ & $O(1)$ \\
    \hline
    BST-Inorder(T) & Visits all nodes in sorted order & $O(n)$ & $O(h)$ \\
    \hline
    BST-Preorder(T) & Visits root before its children & $O(n)$ & $O(h)$ \\
    \hline
    BST-Postorder(T) & Visits children before root & $O(n)$ & $O(h)$ \\
    \hline
    \end{tabular}\\
    \textbf{ Queue Operations}\\
    \begin{tabular}{|p{0.19\linewidth}|p{0.37\linewidth}|p{0.168\linewidth}|p{0.04\linewidth}|}
    \hline
    \textbf{Operation} & \textbf{Description} & \textbf{Time} & \textbf{Space} \\
    \hline
    Construction & Create queue from array of n elements & $O(n)$ & $O(n)$ \\
    \hline
    Queue-Empty(Q) & Returns TRUE if queue is empty, FALSE otherwise & $O(1)$ & $O(1)$ \\
    \hline
    Enqueue(Q, x) & Adds element x to rear of queue Q & $O(1)$ & $O(1)$ \\
    \hline
    Dequeue(Q) & Removes and returns front element from queue Q & $O(1)$ & $O(1)$ \\
    \hline
    \end{tabular}\\
    \textbf{Linked List Operations}\\
    \begin{tabular}{|p{0.19\linewidth}|p{0.37\linewidth}|p{0.168\linewidth}|p{0.04\linewidth}|}
    \hline
    \textbf{Operation} & \textbf{Description} & \textbf{Time} & \textbf{Space} \\
    \hline
    Construction & Create linked list from array of n elements & $O(n)$ & $O(n)$ \\
    \hline
    List-Search(L, k) & Returns pointer to first node with key k, NULL if not found & $O(n)$ & $O(1)$ \\
    \hline
    List-Insert(L, x) & Inserts node x at beginning of list L & $O(1)$ & $O(1)$ \\
    \hline
    List-Delete(L, x) & Removes node x from list L & $O(n)$ find, $O(1)$ del & $O(1)$ \\
    \hline
    \end{tabular}\\
    \end{minipage}
    \end{minipage}
    }
  \end{paracol}
  
  \begin{minipage}[t]{0.32\textwidth}
    \textbf{Matrix Chain Multiplication:}\\
    Find optimal parenthesization to minimize multiplications\\
    \textbf{Bottom-Up (Tabulation):}\\
    1. Create table $m[1...n, 1...n]$ with $m[i,i] = 0$\\
       - $m[i,j]$ stores minimal cost of multiplying matrices $i$ through $j$\\
    2. For $l = 2$ to $n$ (chain length):\\
    3. For $i = 1$ to $n-l+1$:\\
    - Set $j = i+l-1$\\
    - Compute $m[i,j] = \min_{i\leq k<j} \{m[i,k] + m[k+1,j] + p_{i-1}p_kp_j\}$\\
    - Store $k$ in $s[i,j]$ that achieved minimum cost\\
    4. Return $m[1,n]$\\[-15px]
    \algocode{matrix-bottom-up}\vspace*{-25px}
    \textbf{Complexity:}\\
    - Time: $O(n^3)$\\
    - Space: $O(n^2)$ (includes matrix dimensions and tables)\\
    - Table $s[i,j]$ stores index of last matrix in \\first parenthesized group
    \end{minipage}
\end{document} 